\documentclass[11pt]{article}
\usepackage{hyperref}
\usepackage{graphicx}
\usepackage[utf8]{inputenc}

\graphicspath{ {./imgs/} }

\usepackage{makeidx}

\title{Solutions for interview questions}
\author{Piotr Wójcik}
%\date{}

%\makeindex

\begin{document}
\maketitle
\section{Coding question}
The solution is a straightforward application of a breadth-first search algorithm (BFS). For every unprocessed foreground pixel $v$, 
we run a graph traversal from that pixel marking all pixels in a conected component containing $v$. For the purpose of this task, we assume $8$-connectivity.
The solutions takes both $\mathcal{O}(hw)$ time and $\mathcal{O}(hw)$ space, where $h$ and $w$ describe input array height and width respectively. 
The code can be found in my \href{https://github.com/piotrmwojcik/interview_questions}{GitHub repository}. A set of unit tests is also provided.
\section{Data analysis question}
The problem with histologic imaging data used for training of deep-learning models is the heterogenenity of visual appearance (due to the differences in specimen acqusition, staining and scanner calibration) between submitting sites. 
All these factors contribute to the difficulty of generalization of deep learning to the previously unseen data. When a domain shift between data sets exists, machine learning algorithms may be biased towards site-specific visual signatures instead of disease-specific ones. 
Here, we will provide an overview of existing methods for histologic data standarization and assess an optimal strategy for the data set in question.
\begin{figure}
\centering
\includegraphics[scale=0.25]{different_scanners.png}
\caption{Tissue patches ilustrating scanner-induced domain gap}
\end{figure}
\subsection{Assumptions about data sets}
In the problem, we are given three batches of tissue images collected from $3$ different sites and representing $5$ different conditions.
As the histologic examination of hematoxylin and eosin-stained images is a standard in digital pathology, we can assume that 
\begin{thebibliography}{99}
\bibitem{auber21} M.~Aubreville, C.~Bertram, M.~Veta, R.~Klopfleisch, N.~Stathonikos, K.~Breininger, N.~ter~Hoeve, F.~Ciompi, A.~Maier:
 \emph{Quantifying the Scanner-Induced Domain Gap in Mitosis Detection},
arXiv preprint arXiv:2103.16515 (2021)
\end{thebibliography}
\end{document}
